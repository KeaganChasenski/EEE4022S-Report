% ----------------------------------------------------
% Future Work
% ----------------------------------------------------
\documentclass[class=report,11pt,crop=false]{standalone}
\input{../Style/ChapterStyle.tex}
\makenoidxglossaries


\newacronym{fm}{FM}{Frequency Modulation}
\newacronym{am}{AM}{Amplitude Modulation}
\newacronym{em}{EM}{electromagnetic}
\newacronym{iq}{IQ}{In-phase and Quadrature}


\newacronym{dft}{DFT}{Discrete Fourier Transform}
\newacronym{idft}{IDFT}{Inverse Discrete Fourier Transform}
\newacronym{fft}{FFT}{Fast Fourier Transform}
\newacronym{ifft}{IFFT}{Inverse Fast Fourier Transform}

\newacronym{df}{DF}{Direction Finding}
\newacronym{rdf}{RDF}{Radio Direction Finding}
\newacronym{AoA}{AoA}{Angle of Arrival}
\newacronym{rf}{RF}{Radio Frequency}
\newacronym{sdr}{SDR}{Software-Defined Radio}
\newacronym{pd}{PD}{Phase-Difference}
\newacronym{vhf}{VHF}{Very High Frequency}
\newacronym{MHz}{MHz}{Megahertz}
\newacronym{db}{dB}{decibel}
\newacronym{dbm}{dBm}{Decibel-milliwatts}
\newacronym{rx}{Rx}{Receiver}
\newacronym{tx}{Tx}{Transmitter}
\newacronym{dsp}{DSP}{Digital Signal Processing}
\newacronym{vor}{VOR}{Very High Frequency Omnidirection Range}
\newacronym{gps}{GPS}{Global Position System}
\newacronym{adf}{ADF}{Automatic Direction Finders}
\newacronym{ndb}{NDB}{Non-Directional Beacon}
\newacronym{sm}{S meter}{Signal Strength Meter}
\newacronym{tdoa}{TDOA}{Time Difference of Arrival}
\newacronym{ham}{HAM}{an informal name for an amateur radio operator}
\newacronym{wbfm}{WBFM}{Wideband Frequency Modulation}
\newacronym{if}{IF}{Intermediate Frequency}
\newacronym{lp}{LP}{Low Pass}
\newacronym{API}{API}{Application Programming Interface}
\newacronym{fpga}{FPGA}{Field-Programmable Gate Array}
\newacronym{bw}{BW}{Bandwidth}
\newacronym{adc}{ADC}{Analog-to-digital converter}
\newacronym{tv}{Tv}{Television}
\newacronym{ai}{AI}{Artificial Intelligence}
\newacronym{lo}{LO}{Local Oscillator}
\newacronym{icasa}{ICASA}{Independent Communications Authority of South Africa}
\newacronym{usb}{USB}{Universal-serial Buss}
\newacronym{os}{OS}{Operating System}
\newacronym{mimo}{MIMO}{Mutliple input, Multiple output}
\newacronym{vna}{VNA}{Vector Network Analyser}
\newacronym{mse}{MSE}{Mean Squared Error}
\newacronym{SNR}{SNR}{Signal-to-Noise Ratio}

\begin{document}
% ----------------------------------------------------
\chapter{Future Work Recommendations \label{ch:fw}}
\epigraph{``Every new beginning comes from some other beginning’s end.''}%
    {\emph{--- Seneca}}
\vspace{0.5cm}
% ----------------------------------------------------

Due to the short nature and limited scope of this short project, there are  many interesting and important aspects of a \gls{df}, \gls{sdr} system left unexplored. It is hoped that by using the cursory work of this report as a basis for future development, this topic can be developed further. A brief discussion on future workings is provided below. 

\section{DF Antennas Development}
The scope of the project concerned \gls{dsp}, phase extraction and a touch of hardware development. Thus, very little time was spent on the construction of an optimal antenna. The following is a list of suggestions regarding antenna development:

\begin{itemize}
    \item Extension of the project to include at least three antennas.
    \item Further investigation into optimal antenna array geometries.
    \item Tuning of the antenna parameters to better match the resonance frequencies to 150MHz. 
    \item A more efficient ground plane. 
\end{itemize}

\section{DF Hardware Development}
The above mention antenna development leads to the question: how does one connect three or more antennas to a device with only two \gls{rx} channels? Future investigation into this is necessary, but could include:

\begin{itemize}
    \item High-frequency multiplexers.
    \item Antenna switches.
    \item An additional \gls{sdr} (expensive).
\end{itemize}

\newpage
Additional hardware development to \gls{sdr} device includes:
\begin{itemize}
    \item Filtering methods as the frequency band of interest is only 2MHz wide.
    \item Internal register optimisation of the \gls{fpga}.
    \item Optimisation of gain settings.
    \item Further development into the Pi as the host computer. 
    \item Additional LED touch screen as a start. 
\end{itemize}

\section{DF Software Development}
\begin{itemize}
    \item Further implementation of the SoapySDR \gls{API}, removing the need for bloated GNU Radio \textsc{Python} code. 
    \item the \gls{df} system \textsc{Python} code should be ported to a language more suited to high-speed, real-time implementation such as C++. 
    \item the code provided in this project should be further optimised in terms of the \gls{dsp} methods used. 
    \item Additionally the method of connecting to the \gls{sdr}, sampling, writing to a file and subsequent looping should be further optimised.
    \item Further time-alignment methods should be considered.
    \item Internal processing delays within the \gls{sdr} and associated software needs to be considered. 
    \item A real-time display developed for the estimated \gls{AoA}.
\end{itemize}


As a final note, the above recommendations fall into the Wildlife telemetry application suggested in Chapter~\ref{ch:application}.

% ----------------------------------------------------
\ifstandalone
\bibliography{Bibliography/References}
\printnoidxglossary[type=\acronymtype,nonumberlist]
\fi
\end{document}
% ---------------------------------------------------